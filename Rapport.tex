\documentclass[twocolumn]{article}[10pt]
\usepackage[english]{babel} 
\usepackage[latin1]{inputenc} 
\usepackage{times} 			% Default times font style
\usepackage[T1]{fontenc} 	% Font encoding
\usepackage{amsmath} 		% Math package
\usepackage{mathtools} 		% Adds the declare paired 
							% delimeter command to make costom \abs and \norm
\usepackage{breqn}		 	% Adds dmath environment for automated brakeline
\usepackage{xfrac}			% Adds slanted fractions (sfrac)
\usepackage{cancel}			% Adds the cancel command, a slash through the symbol(s)
\usepackage{tabularx}		% Adds adjustable width on tabulars
\usepackage{cuted}			% Adds the strip command, pagewidth text in a twocolumn
							% environment. 
\usepackage{hyperref}
\usepackage{xcolor}


% Start costum \abs \norm 
\DeclarePairedDelimiter\abs{\lvert}{\rvert}%
\DeclarePairedDelimiter\norm{\lVert}{\rVert}%
% Swap the definition of \abs* and \norm*, so that \abs
% and \norm resizes the size of the brackets, and the 
% starred version does not.
\makeatletter
\let\oldabs\abs
\def\abs{\@ifstar{\oldabs}{\oldabs*}}
%
\let\oldnorm\norm
\def\norm{\@ifstar{\oldnorm}{\oldnorm*}}
\makeatother
% End costum \abs \norm 

\title{Project FYS4130}

\begin{document}
\maketitle
{\color{black!70}
Problem1: Calculate $\langle X^2\rangle$ for both correlated and uncorrelated velocities by setting $p=1/2, p=1\%$ and $p=10\%$. Calculate the ensemble average with $10^5$ realizations and use $N = 1000$.
}

In figure ?? we can see the difference between the $p$ values.
Low values of $p$ makes the walker goes further. 

{\color{black!60}2. Make a logarithmic plot, i.e.e plot the variance 
$\log_{10}(\langle\Delta X^2 \rangle))$ as a function of $\log_{10}$ 
with $t = i$ and identify linear portions of the graphs. Discuss the 
corresponding early time exponents of $p=1\%$ and $p=10\%$ graphs and
why they change at later times. 
}

Figure ?? shows $\log_{10}(\langle\Delta X^2 \rangle))$ for the two cases. 

\end{document}